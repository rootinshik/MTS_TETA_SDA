\documentclass[a4paper, 12pt]{article}

\usepackage[utf8]{inputenc}     % Кодировка исходного текста
\usepackage[english, russian]{babel}    % Поддержка русского языка
\usepackage{indentfirst}    % Отступ в первом абзаце
\usepackage{amsmath, amsfonts, amssymb, physics, tabularx}     % Разнообразные команды и значки
\usepackage[f]{esvect}
\usepackage[left=20mm, right=10mm, top=20mm, bottom=20mm]{geometry}     % Отступы
\usepackage[table]{xcolor} % еще таблицы
\usepackage{tabto} % для \tab
\usepackage{minted} % вставка кода
\usepackage{graphicx} 
\usepackage{listings}

\pagenumbering{gobble}  % Отключение нумерации
\newcommand\mymathop[1]{\mathop{\operatorname{#1}}}

\DeclareMathOperator*{\matrixA}{A}
\DeclareMathOperator{\Rank}{Rank}
\DeclareMathOperator{\Dim}{Dim}
\DeclareMathOperator{\Lin}{Lin}
\DeclareMathOperator{\Trig}{Trig}
\DeclareMathOperator{\Leg}{Leg}
\DeclareMathOperator{\Cheb}{Cheb}
\DeclareMathOperator{\Taylor}{Taylor}
\author{Гурьянов Кирилл}

\title{Домашние задание №1 по линейной алгебре. МТС.Teta ШАД}
\date{\today}

\begin{document}

\textbf{Задание №1.} Зная, что $|\Vec{a}| = 2$, $|\Vec{b}|=5$ и $(\widehat{a, b}) = \frac{2 \pi }{3}$, определить, при каком значении коэффициента $\alpha$ векторы $\Vec{p} = \alpha \Vec{a} + 17 \Vec{b}$ и $\Vec{q} = 3\Vec{a} - \Vec{b}$ окажутся перпендикулярными.

\vspace{0.5cm} 

\textbf{Решение задания №1.} Для того, что вектора оказались перпендикулярными необходимо, чтобы скалярное произведение равнялось $0$. То есть: 
\[
(\Vec{p}, \ \Vec{q}) = 0 \ \Rightarrow \ (\alpha \Vec{a} + 17 \Vec{b}, \ 3\Vec{a} - \Vec{b}) = 0
\]
Дважды воспользуемся дистрибутивностью относительно сложения:
\[
 (\alpha \Vec{a} + 17 \Vec{b}, \ 3\Vec{a} - \Vec{b}) = (\alpha \Vec{a}, \ 3\Vec{a} - \Vec{b}) + (17 \Vec{b}, \ 3\Vec{a} - \Vec{b}) = (\alpha \Vec{a}, \ 3\Vec{a}) - (\alpha \Vec{a}, \  \Vec{b}) + (17 \Vec{b}, \ 3\Vec{a}) - (17 \Vec{b}, \ \Vec{b}) = 0
\]
Теперь посчитаем каждое скалярное произведение:
\begin{align*}
    & (\alpha \Vec{a}, \ 3\Vec{a}) = 3 \alpha \cdot (\Vec{a}, \Vec{a}) = 3 \alpha \cdot (\Vec{a}, \Vec{a}) = 3 \alpha \cdot |\Vec{a}| \cdot |\Vec{a}| \cdot \cos(\widehat{a, a}) = 3 \alpha \cdot |\Vec{a}|^2 \cdot 1 = 12 \alpha\\
    & (\alpha \Vec{a}, \  \Vec{b}) = \alpha \cdot (\Vec{a}, \Vec{b}) = \alpha \cdot |\Vec{a}| \cdot |\Vec{b}| \cdot \cos(\widehat{a, b}) = 2 \cdot 5 \cdot \alpha \cdot \left( -\frac{1}{2} \right) = -5\alpha \\  
    & (17 \Vec{b}, \ 3\Vec{a}) = 51 (\Vec{a}, \Vec{b}) = 51 \cdot 2 \cdot 5 \cdot \left( - \frac{1}{2}\right) = -255 \\ 
    & (17 \Vec{b}, \Vec{b}) = 17 \cdot |\Vec{b}|^2 = 17 \cdot 5^2 = 17 \cdot 25 = 425 \\
\end{align*}
Запишем получившиеся уравнение: 
\begin{gather*}
    (\alpha \Vec{a}, \ 3\Vec{a}) - (\alpha \Vec{a}, \  \Vec{b}) + (17 \Vec{b}, \ 3\Vec{a}) - (17 \Vec{b}, \ \Vec{b}) = 0 \\
    12\alpha + 5 \alpha -255 - 425 = 0 \\ 
    \alpha = \frac{255 + 425}{17} \\ 
    \alpha = \frac{680}{17} \\ 
    \alpha = 40
\end{gather*}
\textbf{Ответ:} $\alpha = 40$.
\vspace{1cm}

\textbf{Задание №2.} Пусть $G$ - множество всех векторов-столбцов линейного пространства $\mathbb{R}^2$ с положительными элементами, то есть:
\[
G = \{x: x=
\begin{pmatrix}
    x_1 \\
    x_2 \\ 
    \vdots \\ 
    x_n
\end{pmatrix}
\in \mathbb{R}^n, \ x_i > 0, \ i = 1,2,\dots,n\}.
\]
Проверить, является линейным пространством множество $G$, если операции сложения векторов и умножения вектора на число определяются следующим образом:
\[
x + y =
\begin{pmatrix}
    x_1 \cdot y_1 \\
    x_2 \cdot y_2 \\ 
    \vdots \\
    x_n \cdot y_n 
\end{pmatrix}
\quad \text{и} \quad\ \alpha x = 
\begin{pmatrix}
    x_1^\alpha \\
    x_2^\alpha \\ 
    \vdots \\
    x_n^\alpha 
\end{pmatrix}
\]

\vspace{0.5cm} 

\textbf{Решение задания №2.} Будем проверять аксиомы линейного пространства. Начнем со сложения и его коммутативности ($a + b = b + a$):
\[
a + b = 
\begin{pmatrix}
    a_1 \cdot b_1 \\
    a_2 \cdot b_2 \\ 
    \vdots \\
    a_n \cdot b_n 
\end{pmatrix}
\quad 
b + a = 
\begin{pmatrix}
    b_1 \cdot a_1 \\
    b_2 \cdot a_2 \\ 
    \vdots \\
    b_n \cdot a_n 
\end{pmatrix}
\]
Пользуясь коммутативностью умножения, легко заметить что:
\[
a + b = 
\begin{pmatrix}
    a_1 \cdot b_1 \\
    a_2 \cdot b_2 \\ 
    \vdots \\
    a_n \cdot b_n 
\end{pmatrix}
= \begin{pmatrix}
    b_1 \cdot a_1 \\
    b_2 \cdot a_2 \\ 
    \vdots \\
    b_n \cdot a_n 
\end{pmatrix}
= b + a
\]
Проверим ассоциативность:
\begin{gather*}
(a + b) + c = 
\begin{pmatrix}
    a_1 \cdot b_1 \\
    a_2 \cdot b_2 \\ 
    \vdots \\
    a_n \cdot b_n 
\end{pmatrix}
+ 
\begin{pmatrix}
    c_1 \\ 
    c_2 \\  
    \vdots \\
    c_n  
\end{pmatrix}
=
\begin{pmatrix}
    a_1 \cdot b_1 \cdot c_1 \\
    a_2 \cdot b_2 \cdot c_2 \\ 
    \vdots \\
    a_n \cdot b_n \cdot c_n
\end{pmatrix} 
=
\begin{pmatrix}
    a_1  \\
    a_2  \\ 
    \vdots \\
    a_n  
\end{pmatrix}
+ 
\begin{pmatrix}
    b_1 \cdot c_1 \\ 
    b_2 \cdot c_2 \\  
    \vdots \\
    b_n \cdot c_n 
\end{pmatrix} = 
a + (b + c)
\end{gather*}
Нейтральным элементом в этом поле будет единичный вектор $\mathbf{0} = (1, 1, \dots, 1)^T$. Обратным элементом ($a + (-a) = 0$) будет вектор, состоящий из значений обратных к исходным ($a_n = a_n^{-1}$).
\[
a + (-a) = 
\begin{pmatrix}
    a_1 \\ 
    a_2 \\  
    \vdots \\
    a_n
\end{pmatrix} +
\begin{pmatrix}
    \cfrac{1}{a_1} \\ 
    \cfrac{1}{a_2} \\  
    \vdots \\
    \cfrac{1}{a_n}
\end{pmatrix} = 
\begin{pmatrix}
    \cfrac{a_1}{a_1} \\ 
    \cfrac{a_2}{a_2} \\  
    \vdots \\
    \cfrac{a_n}{a_n}
\end{pmatrix} = 
\begin{pmatrix}
    1 \\ 
    1 \\  
    \vdots \\
    1
\end{pmatrix} = \mathbf{0}
\]
Так как все элементы $a_i > 0$ у нас не возникнет проблемы с делением на $0$. \\ 

Теперь проверим свойства умножения. Коммутативность и ассоциативность очевидна. Нейтральным элементом ($a \cdot 1 = a$) относительно умножения будет $1$:
\[
1 \cdot
\begin{pmatrix}
    a_1 \\ 
    a_2 \\  
    \vdots \\
    a_n
\end{pmatrix}
= 
\begin{pmatrix}
    a_1^1 \\ 
    a_2^1 \\  
    \vdots \\
    a_3^1
\end{pmatrix}
=
\begin{pmatrix}
    a_1 \\ 
    a_2 \\  
    \vdots \\
    a_3
\end{pmatrix}
\]
Обратным элементом для ненулевого элемента будет $0$:
\[
0 \cdot
\begin{pmatrix}
    a_1 \\ 
    a_2 \\  
    \vdots \\
    a_n
\end{pmatrix}
= 
\begin{pmatrix}
    a_1^0 \\ 
    a_2^0 \\  
    \vdots \\
    a_3^0
\end{pmatrix} = 
\begin{pmatrix}
    1 \\ 
    1 \\  
    \vdots \\
    1
\end{pmatrix} = \mathbf{0}
\]
Проверим дистрибутивность умножения на число относительно сложения ($a \cdot (b + c) = a \cdot b + a \cdot c$):
\[
a \cdot (b + c) = a \cdot
\begin{pmatrix}
    b_1 \cdot c_1 \\ 
    b_2 \cdot c_2 \\  
    \vdots \\
    b_n \cdot c_n
\end{pmatrix} = 
\begin{pmatrix}
    (b_1 \cdot c_1)^a \\ 
    (b_2 \cdot c_2)^a \\  
    \vdots \\
    (b_n \cdot c_n)^a
\end{pmatrix} =
\begin{pmatrix}
    b_1^a \cdot c_1^a \\ 
    b_2^a \cdot c_2^a \\  
    \vdots \\
    b_n^a \cdot c_n^a
\end{pmatrix} = 
\begin{pmatrix}
    b_1^a  \\ 
    b_2^a  \\  
    \vdots \\
    b_n^a 
\end{pmatrix} + 
\begin{pmatrix}
    c_1^a  \\ 
    c_2^a  \\  
    \vdots \\
    c_n^a 
\end{pmatrix} = 
a \cdot 
\begin{pmatrix}
    b_1  \\ 
    b_2  \\  
    \vdots \\
    b_n 
\end{pmatrix} + 
a \cdot 
\begin{pmatrix}
    b_1  \\ 
    b_2  \\  
    \vdots \\
    b_n 
\end{pmatrix}
= a \cdot b + a \cdot c
\] 
\vspace{1cm}

\textbf{Задание №3.} Используя определение, доказать, что для любых векторов $x$, $y$, $z$ и чисел $\alpha$, $\beta$, $\gamma$ векторы $\alpha x - \beta y$, $\gamma y - \alpha z$, $\beta z - \gamma x$ линейно зависимы.

\vspace{0.5cm} 

\textbf{Решение задания №3.} Воспользуемся определением линейной зависимости - должен существовать такой набор чисел $a_1, a_2, a_3$ не равных $0$ одновременно, что: $a_1(\alpha x - \beta y) + a_2(\gamma y - \alpha z) + a_3 (\beta z - \gamma x) = 0$.
\begin{gather*}
    a_1(\alpha x - \beta y) + a_2(\gamma y - \alpha z) + a_3 (\beta z - \gamma x) = 0 \\ 
    - \frac{a_1}{a_3}(\alpha x - \beta y) - \frac{a_2}{a_3}(\gamma y - \alpha z) = \beta z - \gamma x \\
    \text{Переобозначим коэффициенты:} \ a_1 = - \frac{a_1}{a_3}, \ a_2 = - \frac{a_2}{a_3}\\
    a_1(\alpha x - \beta y) + a_2(\gamma y - \alpha z) = \beta z - \gamma x \\ 
    \text{Заметим, что при} \ a_1 = -\cfrac{\gamma}{\alpha}, \ a_2 = -\cfrac{\beta}{\alpha} \ \text{левая и правя часть равны} \\ 
    - \cfrac{\gamma}{\alpha} \cdot (\alpha x - \beta y) - \cfrac{\beta}{\alpha} \cdot (\gamma y - \alpha z) = \beta z - \gamma x \\ 
    - \gamma x + \cfrac{\gamma \cdot \beta}{\alpha} y - \cfrac{\beta \cdot \gamma}{\alpha} y + \beta z = \beta z - \gamma x \\
    \beta z - \gamma x = \beta z - \gamma x \\ 
    0 = 0 
\end{gather*}
Таким образом при $a_1 = \cfrac{\gamma}{\alpha}$, $a_2 = \cfrac{\beta}{\alpha}$, $a_3 = 1$ вектор $\beta z - \gamma x$ можно получить из векторов $\alpha x - \beta y$ и $\gamma y - \alpha z$. Таким образом вектора линейно зависимы.

\vspace{1cm}

\textbf{Задание №4.} Проверить, является ли система векторов $e1$, $e2$, $e3$ базисом в линейном пространстве $\mathbb{R}^3$, и найти координаты вектора $x$ в этом базисе. По известному координатному вектору $y_e$ найти вектор $y$;
\[
e_1 = 
\begin{pmatrix}
     -2 \\
     3 \\
     0
\end{pmatrix}, \quad
e_2 = 
\begin{pmatrix}
     2 \\
     -3 \\
     4
\end{pmatrix}, \quad
e_3 = 
\begin{pmatrix}
     -2 \\
     0 \\
     -3
\end{pmatrix}, \quad
x = 
\begin{pmatrix}
     -4 \\
     3 \\
     -7
\end{pmatrix}, \quad
y_e = 
\begin{pmatrix}
     4 \\
     4 \\
     3
\end{pmatrix}
\]

\vspace{0.5cm} 

\textbf{Решение задания №4.} Для того чтобы понять является ли система векторов $e_1$, $e_2$, $e_3$ базисом в пространстве $\mathbb{R}^3$ можно привести матрицу, состоящую из базисных векторов, к треугольному виду. Если полученная матрица не будет содержать нулевых строк, то вектора будут линейно независимы, как следствие, система векторов $e_1$, $e_2$, $e_3$ будет являться базисом в пространстве $\mathbb{R}^3$.
\[
\begin{pmatrix}
     -2 & 2 & -2 \\
     3 & -3 & 0 \\
     0 & 4 & -3
\end{pmatrix} \rightarrow
\begin{pmatrix}
     -2 & 2 & -2 \\
     0 & 0 & -3 \\
     0 & 4 & -3
\end{pmatrix} \rightarrow
\begin{pmatrix}
     -2 & 2 & -2 \\
     0 & 4 & -3 \\ 
     0 & 0 & -3 
\end{pmatrix} 
\]
Полученная матрица не содержит нулевых строк, получаем, что система система векторов $e_1$, $e_2$, $e_3$ является базисом в пространстве $\mathbb{R}^3$. \\ 

Получим координаты вектора $x$ в базисе, состоящим из векторов $e_1$, $e_2$, $e_3$, для этого решим СЛАУ $Ay=x$, где $A$ - матрица из векторов $e_1$, $e_2$, $e_3$, $x$ - вектор, данный по условию, $y$ - вектор $x$ в базисе $e_1$, $e_2$, $e_3$.
\[
\begin{pmatrix}
     -2 & 2 & -2 & \vrule & -4 \\
     3 & -3 & 0 & \vrule & 3 \\
     0 & 4 & -3 & \vrule & -7
\end{pmatrix} \rightarrow 
\begin{pmatrix}
-2 & 2 & -2 & \vrule & -4 \\
0 & 0 & -3 & \vrule & -3 \\
0 & 4 & -3 & \vrule & -7
\end{pmatrix} \rightarrow 
\begin{pmatrix}
-2 & 2 & -2 & \vrule & -4 \\
0 & 4 & -3 & \vrule & -7 \\
0 & 0 & -3 &\vrule & -3
\end{pmatrix}
\]
Получим СЛАУ:
\begin{gather*}
\begin{cases}
    \begin{aligned}
    -2x_1 +2x_2 -2x_3 =& -4 \\
     4x_2 - 3x_3 =& -7 \\
    -3 x_3 =& -3
    \end{aligned}
\end{cases} 
\end{gather*}

Тогда вектор $x$ в новом базисе будет равен $\left(0, -1, 1 \right)$. \\ 

Получим вектор $y$, умножив матрицу $A$ на вектор $y_e$ ($Ay_e = y$):
\[
\begin{pmatrix}
     -2 & 2 & -2 \\
     3 & -3 & 0 \\
     0 & 4 & -3 
\end{pmatrix}
\times
\begin{pmatrix}
     4 \\
     4 \\
     3 
\end{pmatrix} = 
\begin{pmatrix}
    -2\cdot4+2\cdot4+\left(-2\right)\cdot3 \\
    3\cdot4+\left(-3\right)\cdot4+0\cdot3 \\
    0\cdot4+4\cdot4+\left(-3\right)\cdot3
\end{pmatrix} = 
\begin{pmatrix}
     -6 \\
     0 \\
     7 
\end{pmatrix}
\]


\vspace{1cm}

\textbf{Задание №5.} Найти какой-нибудь базис и размерность линейного пространства $V$, заданного следующим образом: 
\[
\text{Пространство многочленов $p(x) \in \mathbb{P}_4$ таких, что $p(1) + p(-1) = 0$}
\]

\vspace{0.5cm} 

\textbf{Решение задания №5.} Распишем условие $p(1) + p(-1) = 0$:
\begin{gather*}
    p(1) + p(-1) = 0 \\ 
    a(1)^4 + b(1)^3 + c(1)^2 + d(1) + e + a(-1)^4 + b(-1)^3 + c(-1)^2 + d(-1) + e = 0 \\ 
    a + b + c + d + e + a - b + c - d + e = 0 \\
    2a + 2c + 2e = 0 \\
    a + c + e = 0 \\
    a = -c - e 
\end{gather*}
Таким образом все многочлены линейного пространства $V$ представимы в виде:
\begin{gather*}
    p(x) = -(c + e)x^4 + bx^3 + cx^2 + dt + e \\ 
    p(x) = -cx^4 - ex^4 + bx^3 + cx^2 + dt + e \\ 
    p(x) = bx^3 + c(x^2 - x^4)+ dt + e(1 - x^4) \\ 
\end{gather*}
Теперь невооруженным глазом виден базис этого линейного пространства:
\begin{align*}
    e_1 = & 1 - x^4 \\ 
    e_2 = & x \\ 
    e_3 = & x^2 - x^4 \\ 
    e_4 = & x^3 
\end{align*}
Соотвественно, размерность этого линейного пространства будет равна $4$.
\end{document}