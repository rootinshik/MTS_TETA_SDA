\documentclass[a4paper, 12pt]{article}

\usepackage[utf8]{inputenc}     % Кодировка исходного текста
\usepackage[english, russian]{babel}    % Поддержка русского языка
\usepackage{indentfirst}    % Отступ в первом абзаце
\usepackage{amsmath, amsfonts, amssymb, physics, tabularx}     % Разнообразные команды и значки
\usepackage[f]{esvect}
\usepackage[left=20mm, right=10mm, top=20mm, bottom=20mm]{geometry}     % Отступы
\usepackage[table]{xcolor} % еще таблицы
\usepackage{tabto} % для \tab
\usepackage{minted} % вставка кода
\usepackage{graphicx} 
\usepackage{listings}

\pagenumbering{gobble}  % Отключение нумерации
\newcommand\mymathop[1]{\mathop{\operatorname{#1}}}

\DeclareMathOperator*{\matrixA}{A}
\DeclareMathOperator{\Rank}{Rank}
\DeclareMathOperator{\Dim}{Dim}
\DeclareMathOperator{\Lin}{Lin}
\DeclareMathOperator{\Trig}{Trig}
\DeclareMathOperator{\Leg}{Leg}
\DeclareMathOperator{\Cheb}{Cheb}
\DeclareMathOperator{\Taylor}{Taylor}

\makeatletter
\renewcommand*\env@matrix[1][\arraystretch]{%
  \edef\arraystretch{#1}%
  \hskip -\arraycolsep
  \let\@ifnextchar\new@ifnextchar
  \array{*\c@MaxMatrixCols c}}
\makeatother

\author{Гурьянов Кирилл}

\title{Домашние задание №2 по линейной алгебре. МТС.Teta ШАД}
\date{\today}

\begin{document}

\textbf{Задание №1.} Найти обратную матрицы методом присоединённой матрицы:
\[
\begin{pmatrix}
    5 & 3 & 1 \\ 
    1 & -3 & -2 \\ 
    -5 & 2 & 1
\end{pmatrix}
\]

\vspace{0.5cm} 

\textbf{Решение задания №1.} Для того, чтобы существовала обратная матрица, необходимо, что она была квадратной (это условие выполнено) и её определитель не был равен $0$. Для начала посчитаем определитель этой матрицы: 
\begin{gather*}
    \begin{vmatrix}
        5 & 3 & 1 \\ 
        1 & -3 & -2 \\ 
        -5 & 2 & 1
    \end{vmatrix} = 5\cdot(-3)\cdot1+3\cdot(-2)\cdot(-5) +\\ 
    +1\cdot1\cdot2-(-5)\cdot(-3)\cdot1-2\cdot(-2)\cdot5-1\cdot1\cdot3 = 19
\end{gather*}
Теперь найдем обратную метолом присоединенной матрицы:
\[
A^{-1} = \frac{1}{|A|} \cdot \overline{A}, \quad
\overline{A} = 
\begin{pmatrix}
    \overline{a}_{11} & \overline{a}_{12} & \overline{a}_{13} \\ 
    \overline{a}_{21} & \overline{a}_{22} & \overline{a}_{23} \\ 
    \overline{a}_{31} & \overline{a}_{32} & \overline{a}_{33}
\end{pmatrix}
\]
Пример расчетов для $a_{11}$ и $a_{12}$:
\begin{align*}
    & a_{11} = (-1)^{1 + 1} \cdot 
    \begin{vmatrix}
        -3 & -2 \\
        2 & 1
    \end{vmatrix} = 1 \cdot (-3 \cdot 1 - (-2) \cdot 2) = 1\\ 
    & a_{12} = (-1)^{1 + 2} \cdot 
    \begin{vmatrix}
        1 & -2 \\
        -5 & 1
    \end{vmatrix} = -1 \cdot (1 \cdot 1 - (-2) \cdot (-5)) = 9
\end{align*}
Посчитаем все $\overline{a}_{ij}$:
\[
\overline{A} = 
\begin{pmatrix}
    1 & 9 & -13  \\
    -1 & 10 & -25 \\ 
    -3 & 11 & -18
\end{pmatrix}
\]
Продолжим вычисление обратной матрицы:
\[
A^{-1} = \frac{1}{|A|} \cdot \overline{A} = \frac{1}{19} \cdot
\begin{pmatrix}
    1 & 9 & -13  \\
    -1 & 10 & -25 \\ 
    -3 & 11 & -18
\end{pmatrix} = 
\begin{pmatrix}[1.25]
    \frac{1}{19} & \frac{-1}{19} & \frac{-3}{19}  \\ 
    \frac{9}{19} & \frac{10}{19} & \frac{11}{19}  \\
    \frac{-13}{19} & \frac{-25}{19} & \frac{-18}{19}
\end{pmatrix}
\]

\newpage

\textbf{Задание №2.} Решить матричное уравнение:
\[
\begin{pmatrix}
    1 & -2 & 3 \\ 
    2 & 3 & -1 \\ 
    0 & -2 & 1
\end{pmatrix}
\cdot X \cdot 
\begin{pmatrix}
    1 & 2 & 3 \\ 
    4 & 5 & 6 \\ 
    7 & 8 & 0
\end{pmatrix} = 
\begin{pmatrix}
    1 & 2 & 3 \\ 
    4 & 5 & 6 \\ 
    7 & 8 & 0
\end{pmatrix}
\]

\vspace{0.5cm} 

\textbf{Решение задания №2.} Выразим матрицу $X$:
\begin{gather*}
    \begin{pmatrix}
    1 & -2 & 3 \\ 
    2 & 3 & -1 \\ 
    0 & -2 & 1
\end{pmatrix}
\cdot X \cdot 
\begin{pmatrix}
    1 & 2 & 3 \\ 
    4 & 5 & 6 \\ 
    7 & 8 & 0
\end{pmatrix} = 
\begin{pmatrix}
    1 & 2 & 3 \\ 
    4 & 5 & 6 \\ 
    7 & 8 & 0
\end{pmatrix} \\ 
\begin{pmatrix}
    1 & -2 & 3 \\ 
    2 & 3 & -1 \\ 
    0 & -2 & 1
\end{pmatrix}
\cdot X =  
\begin{pmatrix}
    1 & 2 & 3 \\ 
    4 & 5 & 6 \\ 
    7 & 8 & 0
\end{pmatrix} \cdot 
\begin{pmatrix}
    1 & 2 & 3 \\ 
    4 & 5 & 6 \\ 
    7 & 8 & 0
\end{pmatrix}^{-1} \\ 
\begin{pmatrix}
    1 & -2 & 3 \\ 
    2 & 3 & -1 \\ 
    0 & -2 & 1
\end{pmatrix}
\cdot X =  
E \\ 
X = \begin{pmatrix}
    1 & -2 & 3 \\ 
    2 & 3 & -1 \\ 
    0 & -2 & 1
\end{pmatrix} ^{-1}
\cdot E \\ 
X = \begin{pmatrix}
    1 & -2 & 3 \\ 
    2 & 3 & -1 \\ 
    0 & -2 & 1
\end{pmatrix} ^{-1}
\end{gather*}
Для этого решения необходима обратимость всех матриц, проверим это. Они квадратные, так что осталось только проверить определитель:
\begin{gather*}
    \begin{vmatrix}
        1 & -2 & 3 \\ 
        2 & 3 & -1 \\ 
        0 & -2 & 1
    \end{vmatrix} = 1\cdot3\cdot1+(-2)\cdot(-1)\cdot0+3\cdot2\cdot(-2)- \\ 
    - 0\cdot3\cdot3-(-2)\cdot(-1)\cdot1-1\cdot2\cdot(-2) = -7 
\end{gather*}
\begin{gather*}
    \begin{vmatrix}
        1 & 2 & 3 \\ 
        4 & 5 & 6 \\ 
        7 & 8 & 0
    \end{vmatrix} = 1\cdot5\cdot0+2\cdot6\cdot7+3\cdot4\cdot8-7\cdot5\cdot3- \\
    -8\cdot6\cdot1-0\cdot4\cdot2 = 27  
\end{gather*}
Обе матрицы обратимы. Посчитаем обратную матрицу (обозначим её за $A$) для нахождения решения матричного уравнения с помощью метода присоединенной матрицы:
\[
\overline{A} = 
\begin{pmatrix}
    1 & -4 & -7 \\
    -2 & 1 & 7 \\
    -4 & 2 & 7
\end{pmatrix}, \quad  A^{-1} = \frac{1}{|A|} \cdot \overline{A} = 
\begin{pmatrix}[1.5]
    \frac{-1}{7} & \frac{4}{7} & 1 \\
    \frac{2}{7} & \frac{-1}{7} & -1 \\
    \frac{4}{7} & \frac{-2}{7} & -1
\end{pmatrix}
\]
Тогда $X=A^{-1}$:
\[
X = 
\begin{pmatrix}[1.25]
    \frac{-1}{7} & \frac{4}{7} & 1 \\
    \frac{2}{7} & \frac{-1}{7} & -1 \\
    \frac{4}{7} & \frac{-2}{7} & -1
\end{pmatrix}
\]

\newpage

\textbf{Задание №3.} Найти ранг матрицы при различных значениях параметра:
\[
\begin{pmatrix}
    3 & 1 & 1 & 4 \\ 
    \lambda & 4 & 10 & 1 \\ 
    1 & 7 & 17 & 3 \\
    2 & 2 & 4 & 3 
\end{pmatrix}
\]

\vspace{0.5cm}

\textbf{Решение задания №3.}
\begin{gather*}
    \begin{pmatrix}
        3 & 1 & 1 & 4 \\ 
        \lambda & 4 & 10 & 1 \\ 
        1 & 7 & 17 & 3 \\
        2 & 2 & 4 & 3 
    \end{pmatrix} \rightarrow
    \begin{pmatrix}[1.25]
        3 & 1 & 1 & 4 \\
        0 & \frac{-\lambda+12}{3} & \frac{-\lambda+30}{3} & \frac{-4\cdot\lambda+3}{3} \\
        1 & 7 & 17 & 3 \\
        2 & 2 & 4 & 3
    \end{pmatrix} \rightarrow
    \begin{pmatrix}[1.25]
        3 & 1 & 1 & 4 \\
        0 & \frac{-\lambda+12}{3} & \frac{-\lambda+30}{3} & \frac{-4\cdot\lambda+3}{3} \\
        0 & \frac{20}{3} & \frac{50}{3} & \frac{5}{3} \\
        2 & 2 & 4 & 3
    \end{pmatrix} \rightarrow \\ \rightarrow
    \begin{pmatrix}[1.25]
       3 & 1 & 1 & 4 \\
       0 & \frac{-\lambda+12}{3} & \frac{-\lambda+30}{3} & \frac{-4\cdot\lambda+3}{3} \\
       0 & \frac{20}{3} & \frac{50}{3} & \frac{5}{3} \\
       0 & \frac{4}{3} & \frac{10}{3} & \frac{1}{3}
    \end{pmatrix} \rightarrow
    \begin{pmatrix}[1.25]
        3 & 1 & 1 & 4 \\
        0 & \frac{20}{3} & \frac{50}{3} & \frac{5}{3} \\
        0 & \frac{-\lambda+12}{3} & \frac{-\lambda+30}{3} & \frac{-4\cdot\lambda+3}{3} \\
        0 & \frac{4}{3} & \frac{10}{3} & \frac{1}{3}
    \end{pmatrix} \rightarrow
    \begin{pmatrix}[1.25]
        3 & 1 & 1 & 4 \\
        0 & \frac{20}{3} & \frac{50}{3} & \frac{5}{3} \\
        0 & 0 & \frac{\lambda}{2} & \frac{-5\cdot\lambda}{4} \\
        0 & \frac{4}{3} & \frac{10}{3} & \frac{1}{3}
    \end{pmatrix} \rightarrow \\ \rightarrow 
    \begin{pmatrix}[1.25]
        3 & 1 & 1 & 4 \\
        0 & \frac{20}{3} & \frac{50}{3} & \frac{5}{3} \\
        0 & 0 & \frac{\lambda}{2} & \frac{-5\cdot\lambda}{4} \\
        0 & 0 & 0 & 0
    \end{pmatrix}
\end{gather*}
Получаем, что при $\lambda \ne 0$ ранг нашей матрицы равен $3$, при $\lambda = 0$ ранг 2. 

\newpage

\textbf{Задание №4.} Исследовать систему и найти решение в зависимости от значения параметра:
\[
\begin{cases}
    x_1 + 4x_2 + 2x_3 = -1 \\ 
    2x_1 + 3x_2 - x_3 = 3 \\ 
    x_1 - x_2 - 3x_3 = 4 \\ 
    x_1 - 6x_2 - \lambda x_3 = 9
\end{cases}
\]

\vspace{0.5cm}

\textbf{Решение задания №4.} Перепишем систему в матричный вид, чтобы применить метод Гаусса, и найдем решение системы в зависимости от параметра.
\begin{gather*}
    \begin{pmatrix}
        1 & 4 & 2 & \vrule & -1 \\ 
        2 & 3 & -1 & \vrule & 3 \\ 
        1 & -1 & -3 & \vrule & 4 \\ 
        1 & -6 & -\lambda & \vrule & 9 \\ 
    \end{pmatrix} \rightarrow
    \begin{pmatrix}
        1 & 4 & 2 & \vrule & -1 \\
        0 & -5 & -5 & \vrule & 5 \\
        1 & -1 & -3 & \vrule & 4 \\
        1 & -6 & -\lambda & \vrule & 9 
    \end{pmatrix} \rightarrow
    \begin{pmatrix}
        1 & 4 & 2 & \vrule & -1 \\
        0 & -5 & -5 & \vrule & 5 \\
        0 & -5 & -5 & \vrule & 5 \\
        1 & -6 & -\lambda & \vrule & 9
    \end{pmatrix} \rightarrow \\ \rightarrow
    \begin{pmatrix}
        1 & 4 & 2 & \vrule & -1 \\
        0 & -5 & -5 & \vrule & 5 \\
        0 & -5 & -5 & \vrule & 5 \\
        0 & -10 & -\lambda-2 & \vrule & 10
    \end{pmatrix} \rightarrow
    \begin{pmatrix}
        1 & 4 & 2 & \vrule & -1 \\
        0 & -5 & -5 & \vrule & 5 \\
        0 & 0 & 0 & \vrule & 0 \\
        0 & -10 & -\lambda-2 & \vrule & 10
    \end{pmatrix} \rightarrow 
    \begin{pmatrix}
        1 & 4 & 2 & -1 \\
        0 & -5 & -5 & 5 \\
        0 & 0 & -\lambda + 8 & 0 \\ 
        0 & 0 & 0 & 0 
    \end{pmatrix}
\end{gather*}
При $\lambda \ne 8$, когда $\Rank{(A|b)} = \Rank{(A)}$, у нас одно решение:
\[
\begin{cases}
    x_1 + 4x_2 + 2x_3 = -1 \\ 
    -5x_2 - 5x_3 = 5 \\ 
    (-\lambda + 8)x_3 = 0 \\ 
\end{cases} \quad x = (3, \ -1, \ 0)
\]
При $\lambda = 8$, когда $\Rank{(A|b)} \ne \Rank{(A)}$, у нас бесконечно много решений:
\begin{gather*}    
    \begin{cases}
        x_1 + 4x_2 + 2x_3 = -1 \\ 
        -5x_2 - 5x_3 = 5 
    \end{cases} \\ 
    X = 
    \begin{pmatrix}
        3 \\ 
        -1 \\ 
        0
    \end{pmatrix} +
    \begin{pmatrix}
        2c \\ 
        -c \\ 
        c
    \end{pmatrix}, \quad c \in \mathbb{R}
\end{gather*}

\newpage 

\textbf{Задание №5.} Найти собственные значения и вектора для данной матрицы:
\[
\begin{pmatrix}
    1 & -3 & 4 \\ 
    4 & -7 & 8 \\ 
    6 & -7 & 7
\end{pmatrix}
\]

\vspace{0.5cm}

\textbf{Решение задания №5.} Для нахождения собственных значений, а потом и собственных векторов, необходимо решить характеристическое уравнение:
\begin{gather*}
    |A - \lambda E| = 
    \begin{vmatrix}
        1 - \lambda & -3 & 4 \\ 
        4 & -7 - \lambda & 8 \\ 
        6 & -7 & 7 - \lambda
    \end{vmatrix} = (-\lambda+1)\cdot(-\lambda-7)\cdot(-\lambda+7)+(-3)\cdot8\cdot6+4\cdot4\cdot(-7)- \\ 
    -6\cdot(-\lambda-7)\cdot4-(-7)\cdot8\cdot(-\lambda+1)-(-\lambda+7)\cdot4\cdot(-3) = \\
    = -\lambda^3+\lambda^2+5\lambda +3 = 0
\end{gather*}
Корни многочлена находится среди делителей свободного члена. Заметим, что при $\lambda=3$ Получаем верное равенство. Поделим многочлен на $\lambda-3$. Получим следующий многочлен:
\begin{gather*}
    (\lambda - 3)(-a^2 -2a -1) = 0 \\ 
    (\lambda - 3)(a^2 +2a +1) = 0 \\
    (\lambda - 3)(a+1)^2 = 0 \\ 
    \lambda_1 = 3, \quad \lambda_{2,3} = -1
\end{gather*}
Собственные значения равны $\lambda_1 = 3$ с кратностью $1$ и $\lambda_2 = -1$ с кратностью $2$. Теперь найдем собственный вектор для $\lambda_1 = 3$:
\begin{gather*}
    \begin{pmatrix}
        1 - 3 & -3 & 4 & \vrule & 0 \\ 
        4 & -7 - 3 & 8 & \vrule & 0 \\ 
        6 & -7 & 7 - 3 & \vrule & 0
    \end{pmatrix} \rightarrow
    \begin{pmatrix}
        -2 & -3 & 4 & \vrule & 0 \\ 
        4 & -10 & 8 & \vrule & 0 \\ 
        6 & -7 & 4 & \vrule & 0
    \end{pmatrix} \rightarrow
    \begin{pmatrix}
        -2 & -3 & 4 & \vrule & 0 \\
        0 & -16 & 16 & \vrule & 0 \\
        6 & -7 & 4 & \vrule & 0
    \end{pmatrix} \rightarrow \\ \rightarrow  
    \begin{pmatrix}
        -2 & -3 & 4 & \vrule & 0 \\
        0 & -16 & 16 & \vrule & 0 \\
        0 & -16 & 16 & \vrule & 0
    \end{pmatrix} \rightarrow
    \begin{pmatrix}
        -2 & -3 & 4 & \vrule & 0 \\
        0 & -16 & 16 & \vrule & 0 \\
        0 & 0 & 0 & \vrule & 0
    \end{pmatrix}
\end{gather*}
Получаем систему:
\[
\begin{cases}
        -2x_1 -3x_2 + 4x_3 = 0 \\ 
        -16x_2 - 16x_3 = 0 
\end{cases} 
\]
Её решение:
\[
x = c \cdot
\begin{pmatrix}
    1 \\
    2 \\
    2 \\
\end{pmatrix}, \quad c \in \mathbb{R}
\]
Для собственного числа $\lambda_{2,3} = -1$:
\begin{gather*}
    \begin{pmatrix}
        1 +1 & -3 & 4 & \vrule & 0 \\ 
        4 & -7 +1 & 8 & \vrule & 0 \\ 
        6 & -7 & 7 + 1 & \vrule & 0
    \end{pmatrix} \rightarrow
    \begin{pmatrix}
        2 & -3 & 4 & \vrule & 0 \\ 
        4 & -6 & 8 & \vrule & 0 \\ 
        6 & -7 & 8 & \vrule & 0
    \end{pmatrix} \rightarrow
    \begin{pmatrix}
        2 & -3 & 4 & \vrule & 0 \\
        0 & 0 & 0 & \vrule & 0 \\
        6 & -7 & 8 & \vrule & 0
    \end{pmatrix} \rightarrow \\ \rightarrow  
    \begin{pmatrix}
        2 & -3 & 4 & \vrule & 0 \\
        0 & 0 & 0 & \vrule & 0 \\
        0 & 2 & -4 & \vrule & 0
    \end{pmatrix} \rightarrow
    \begin{pmatrix}
        2 & -3 & 4 & \vrule & 0 \\
        0 & 2 & -4 & \vrule & 0 \\
        0 & 0 & 0 & \vrule & 0
    \end{pmatrix}
\end{gather*}
Получаем систему:
\[
\begin{cases}
        2x_1 -3x_2 + 4x_3 = 0 \\ 
        2x_2 - 4x_3 = 0 
\end{cases} 
\]
Её решение:
\[
x = c \cdot
\begin{pmatrix}
    1 \\
    2 \\
    1 \\
\end{pmatrix}, \quad c \in \mathbb{R}
\]
\end{document}