\documentclass[a4paper, 12pt]{article}

\usepackage[utf8]{inputenc}     % Кодировка исходного текста
\usepackage[english, russian]{babel}    % Поддержка русского языка
\usepackage{indentfirst}    % Отступ в первом абзаце
\usepackage{amsmath, amsfonts, amssymb, physics, tabularx}     % Разнообразные команды и значки
\usepackage[f]{esvect}
\usepackage[left=20mm, right=10mm, top=20mm, bottom=20mm]{geometry}     % Отступы
\usepackage[table]{xcolor} % еще таблицы
\usepackage{tabto} % для \tab
\usepackage{minted} % вставка кода
\usepackage{graphicx} 
\usepackage{listings}

\pagenumbering{gobble}  % Отключение нумерации
\newcommand\mymathop[1]{\mathop{\operatorname{#1}}}
\newcommand{\Mod}[1]{\ (\mathrm{mod}\ #1)}

\DeclareMathOperator*{\matrixA}{A}
\DeclareMathOperator{\Rank}{Rank}
\DeclareMathOperator{\Dim}{Dim}
\DeclareMathOperator{\Lin}{Lin}
\DeclareMathOperator{\Trig}{Trig}
\DeclareMathOperator{\Leg}{Leg}
\DeclareMathOperator{\Cheb}{Cheb}
\DeclareMathOperator{\Taylor}{Taylor}

\makeatletter
\renewcommand*\env@matrix[1][\arraystretch]{%
  \edef\arraystretch{#1}%
  \hskip -\arraycolsep
  \let\@ifnextchar\new@ifnextchar
  \array{*\c@MaxMatrixCols c}}
\makeatother

\author{Гурьянов Кирилл}

\title{Домашние задание №1 по статистике. МТС.Teta ШАД}
\date{\today}

\begin{document}

\textbf{Задание №1.} Компания по страхованию автомобилей разделяет водителей по $3$ классам: класс А (мало рискует), класс В (рискует средне), класс С (рискует сильно). Компания предполагает, что из всех водителей, застрахованных у неё, $30\%$ принадлежат классу А, $50\%$ – классу В, $20\%$ – классу С. Вероятность того, что в течение года водитель класса А попадёт хотя бы в одну автокатастрофу, равна $0,01$ для водителя класса В эта вероятность равна $0,03$, а для водителя класса С – $0,1$. Мистер Джонс страхует свою машину у этой компании и в течение года попадает в автокатастрофу. Какова вероятность того, что он относится к классу А?


\vspace{0.5cm} 

\textbf{Решение задания №1.} Нам нужно найти вероятность того, что водитель принадлежит классу A, при условии, что он попал в аварию. Воспользуемся теоремой Байеса (для нахождения искомой вероятности) и формулой полной вероятности (для нахождения вероятности попадания в аварию):
\begin{gather*}
    \mathbb{\mathbb{P}}(\text{Класс водителя = A} | \text{Водитель попал в аварию}) =  \\ 
    = \frac{\mathbb{\mathbb{P}}(\text{Водитель попал в аварию} | \text{Класс водителя = A}) \cdot \mathbb{\mathbb{P}}(\text{Класс водителя = A})}{\mathbb{\mathbb{P}}(\text{Водитель попал в аварию})} = \\
\end{gather*}
Воспользуемся формулой полной вероятности для нахождения $\mathbb{P}(\text{Водитель попал в аварию})$:
\begin{align*}
    &\mathbb{P}(\text{Водитель попал в аварию}) = \\ 
        &\mathbb{P}(\text{Водитель попал в аварию} | \text{Класс водителя = A}) \cdot \mathbb{P}(\text{Класс водителя = A}) + \\
        + &\mathbb{P}(\text{Водитель попал в аварию} | \text{Класс водителя = A}) \cdot \mathbb{P}(\text{Класс водителя = A}) + \\ 
        + &\mathbb{P}(\text{Водитель попал в аварию} | \text{Класс водителя = A}) \cdot \mathbb{P}(\text{Класс водителя = A})
\end{align*}
Окончательные расчеты:
\[
    \frac{0.01 \cdot 0.3}{0.3 \cdot 0.01 + 0.5 \cdot 0.03 + 0.2 \cdot 0.1} = \frac{3}{38} \approx 0.079
\]

\textbf{Ответ: } $\mathbb{P}(\text{Класс водителя = A} | \text{Водитель попал в аварию}) = \cfrac{3}{38} \approx 0.079$.

\newpage

\textbf{Задание №2.} Движением частицы по целым точкам прямой управляет схема Бернулли с вероятностью $p$ исхода $1$. Если в данном испытании схемы Бернулли появилась $1$, то частица из своего положения переходит в правую соседнюю точку, а в противном случае -- в левую. Найти вероятность того, что за $n$ шагов частица из точки $0$ перейдет в точку $m$.


\vspace{0.5cm} 

\textbf{Решение задания №2.} Рассмотрим случай, когда $m \ge 0$. Для того, чтобы частица перешла в нужную нам точку необходимо, чтобы она в сумме совершила $n - \frac{n - m}{2}$ шагов вправо и $\frac{n - m}{2}$ шагов влево. Если $n \not \equiv m \Mod{2}$, тогда точка недостижима (нельзя попасть в точку $2$ за $3$ шага):
\[
    \mathbb{P}(x = m) = 
    \begin{cases}
        C_n^{\frac{n - m}{2}} p^{n - \frac{n - m}{2}} (1 - p)^{\frac{n - m}{2}}, & n \equiv m \Mod{2} \\ 
        0, & n \not \equiv m \Mod{2}
    \end{cases}
\]
Рассмотрим случай, когда точка $m$ отрицательная и когда $m > n$. В случае с отрицательной точкой $m$ нужно <<шагать>> в противоположную сторону, в случае $m > n$ вероятность будет равна $0$.

\textbf{Ответ:} 
$
\mathbb{P}(x = m) = 
    \begin{cases}
        C_n^{\frac{n - m}{2}} p^{n - \frac{n - m}{2}} (1 - p)^{\frac{n - m}{2}}, & n \equiv m \Mod{2} \text{ и } 0 \le m \le n \\ 
        C_n^{\frac{n - \abs{m}}{2}} p^{\frac{n - \abs{m}}{2}} (1 - p)^{n - \frac{n - \abs{m}}{2}}, & n \equiv m \Mod{2} \text{ и } -n \le m < 0 \\ 
        0, & n \not \equiv m \Mod{2} \text{ или } \abs{m} > n
    \end{cases}
$

\newpage

\textbf{Задание №3.} Плотность распределения $p(x)$ некоторой случайно величины имеет вид $p(x) = \cfrac{c}{e^x + e^{-x}}$ , где $c$ -- константа. Найти значение этой константы $c$ и вероятность того, что случайная величина примет значение, принадлежащее интервалу $( \pi; \pi)$.
\vspace{0.5cm}

\textbf{Решение задания №3.} Для того, чтобы существовала наша функция плотности вероятности, необходимо, чтобы следующий определенный интеграл сходился к 1:
\begin{gather*}
    \int_{-\infty}^{+\infty} p(x) dx = 1 \\ 
    \int_{-\infty}^{+\infty} \frac{c}{e^x + e^{-x}} dx = 1 \\ 
\end{gather*}
Найдем константу $c$:
\begin{gather*}
    \int_{-\infty}^{+\infty} \frac{dx}{e^x + e^{-x}} = \frac{1}{c} \\ 
    \int_{-\infty}^{+\infty} \frac{dx}{e^x + \frac{1}{e^x}} = \frac{1}{c} \\ 
    \int_{-\infty}^{+\infty} \frac{dx}{\frac{e^{2x}}{e^x} + \frac{1}{e^x}} = \frac{1}{c} \\ 
    \int_{-\infty}^{+\infty} \frac{dx}{\frac{e^{2x} + 1}{e^x}} = \frac{1}{c} \\
    \int_{-\infty}^{+\infty} \frac{e^x}{e^{2x} + 1} dx = \frac{1}{c}
\end{gather*}
Сделаем замену переменной $u = e^x$, тогда $du = e^x dx$ и $dx = \cfrac{dx}{e^x}$:
\[
    \int_{0}^{+\infty} \frac{du}{u^2 + 1} = \frac{1}{c}
\]
Получаем табличный интеграл:
\begin{gather*}
    \int_{0}^{+\infty} \frac{du}{u^2 + 1} = \arctan(u) \Big |_{0}^{+\infty} = \frac{\pi}{2} = \frac{1}{c} \\ 
    \frac{\pi}{2} = \frac{1}{c} \\ 
    c = \frac{2}{\pi}
\end{gather*}

Для того чтобы узнать вероятность того, что случайная величина принимает значение из интервала $(-\pi; \pi)$, можно воспользоваться определением плотности вероятности:
\[
    \mathbb{P}(x \in [-\pi; \pi]) = \int_{-\pi}^{\pi} \frac{\frac{2}{\pi}}{e^x + e^{-x}} dx = \frac{2}{\pi} \int_{-\pi}^{\pi} \frac{dx}{e^x + e^{-x}}
\]
Далее посчитаем этот интеграл, который похож на тот, который мы уже считали:
\[
    \frac{2}{\pi} \int_{-\pi}^{\pi} \frac{dx}{e^x + e^{-x}} = \frac{2}{\pi} \arctan{(e^x)} \Big |_{-\pi}^{\pi} = \frac{2}{\pi} (\arctan{(e^{\pi})} - \arctan{(e^{-\pi})}) \approx 0.945
\]


\textbf{Ответ:} $c = \cfrac{2}{\pi}$, $\mathbb{P}(x \in [-\pi, \pi]) \approx 0.945$.

\newpage

\textbf{Задание №4.} Задана плотность распределения вероятностей $f(x)$ непрерывной случайно величины $x$:
\[
f(x) = 
\begin{cases}
    A \sqrt{x}, & x \in [1; 4] \\ 
    0, & x \not \in [1; 4]
\end{cases}
\]
Найти функцию распределения $F(x)$ и $\mathbb{P}(2 < X < 3)$.
\vspace{0.5cm}

\textbf{Решение задания №4.} Найдем значение параметра $A$:
\begin{gather*}
    \int_{-\infty}^{+\infty} p(x)  dx = 1 \\ 
    \int_{1}^{4} A \sqrt{x} dx = 1 \\ 
    \int_{1}^{4} \sqrt{x} dx = \frac{1}{A}
\end{gather*}
Отдельно посчитаем этот интеграл:
\[
    \int_{1}^{4} \sqrt{x} dx = \int_{1}^{4} x^{\frac{1}{2}} dx = \left( \frac{2}{3} \sqrt{x^3} \right) \Bigg |_1^4 = \frac{2}{3} (8 - 1) = \frac{14}{3}
\]
Тогда наш параметр $A = \frac{3}{14}$. Функция плотности вероятности принимает вид:
\[
f(x) = 
\begin{cases}
    \cfrac{3}{14} \sqrt{x}, & x \in [1; 4] \\ 
    0, & x \not \in [1; 4]
\end{cases}
\]
Теперь мы можем найти функцию распределения по определению:
\begin{gather*}
    \int_{-\infty}^x f(x) = F(x) \\ 
    \int_{1}^x \frac{3}{14} \sqrt{x} dx = \frac{3}{14} \left( \frac{2}{3} \sqrt{x^3} - \frac{2}{3} \right) = \frac{1}{7} \sqrt{x^3} - \frac{1}{7}
\end{gather*}
Тогда:
\[
F(x) = 
\begin{cases}
    0, & x < 1 \\
    \frac{1}{7} \sqrt{x^3} - \frac{1}{7}, & 1 \le x \le 4 \\ 
    1, & x > 4
\end{cases}
\]

Теперь найдем $\mathbb{P}(2 < x < 3)$:
\[
    \mathbb{P}(2 < X < 3) = F(3) - F(2) =  \frac{1}{7} \sqrt{3^3} - \frac{1}{7} -  \frac{1}{7} \sqrt{2^3} + \frac{1}{7} = \frac{1}{7} \left( \sqrt{3^3} - \sqrt{2^3} \right) \approx 0.338
\] 

\textbf{Ответ: } $\mathbb{P}(2 < X < 3) \approx 0.338$, 
$F(x) = 
\begin{cases}
    0, & x < 1 \\
    \frac{1}{7} \sqrt{x^3} - \frac{1}{7}, & 1 \le x \le 4 \\ 
    1, & x > 4
\end{cases}.
$

\newpage 

\textbf{Задание №5.} При работе ЭВМ время от времени возникают сбои. Поток сбоев можно считать простейшим. Среднее число сбоев за сутки равно 1.5. Найти вероятность того, что в течение суток произойдет хотя бы один сбой.

\vspace{0.5cm}

\textbf{Решение задания №5.} Воспользуемся распределением Пуассона.
\[
    p(k) \equiv \mathbb{P}(Y = k) = \frac{\lambda ^ k}{k!}e^{-\lambda}
\]
$k$ - это количество событий, $\lambda$ - математическое ожидание (среднее значение). Тогда наша искомая вероятность будет выражаться следующим образом:
\begin{gather*}
    p(\text{произойдет хотя бы 1 сбой}) = 1 - p(\text{произойдет 0 сбоев}) = \\ 
    = 1 - p(0) = 1 - \frac{1.5 ^ 0}{0!}e^{-1.5} = 1 - e^{-1.5} \approx 0.777
\end{gather*}


\textbf{Ответ:} $p(\text{произойдет хотя бы 1 сбой}) \approx 0.777$.
\end{document}